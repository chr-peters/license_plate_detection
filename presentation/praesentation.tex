\documentclass[10pt]{beamer}

\usetheme[subsectionpage=progressbar]{metropolis}
\usepackage{appendixnumberbeamer}

\usepackage{ucs}
\usepackage[utf8x]{inputenc}
\usepackage[ngerman]{babel}

\usepackage{booktabs}
\usepackage[scale=2]{ccicons}

\usepackage{pgfplots}
\usepgfplotslibrary{dateplot}

\usepackage{xspace}
\usepackage{dsfont}
\usepackage{graphicx}
\newcommand{\themename}{\textbf{\textsc{metropolis}}\xspace}

\title{Nummernschilderkennung mit Python}
% \subtitle{Machine Learning in Produktion und Logistik}
\date{19. Januar 2021}
\author{Anne-Sophie Bollmann, Susanne Kl\"ocker, Pia von Kolken, Christian Peters}
%\institute{TU Dortmund}
%\titlegraphic{\hfill\includegraphics[height=1.5cm]{logo.pdf}}

\begin{document}

\maketitle

\begin{frame}{Inhalt}
  \setbeamertemplate{section in toc}[sections numbered]
  \tableofcontents
\end{frame}

\section{Einleitung}

\begin{frame}{Einleitung}
    Ziel: Erkennen von Nummernschildern auf Fotos und Auslesen der Nummernschilder
    
    Herausforderungen:
    \begin{itemize}
        \item Vielfältigkeit der Nummernschilder
        \item Rahmenbedingungen der Bildaufnahme (Beleuchtung)
    \end{itemize}
\end{frame}

\begin{frame}{Beispiel}
  \begin{figure}
  \begin{center}
   \includegraphics{Bild2}[scale=0.25]
   \caption{Original}
   \label{Original}
\end{center}
\end{figure}
 
\end{frame}


\section{Extraktion des Nummernschildes}

\begin{frame}{Convolutional Neural Networks}
    \begin{figure}
        \includegraphics[width=\textwidth]{bilder/Typical_cnn.png}
        \caption{Convolutional Neural Network.
            \footnote{Bildquelle: \url{https://de.wikipedia.org/wiki/Convolutional_Neural_Network}}}
    \end{figure}
    \begin{center}
        \fbox{\textbf{Input:} Bild mit Auto $\longmapsto$ \textbf{Output:} Bounding Box}
    \end{center}
\end{frame}

\begin{frame}{Implementierung}
    \textbf{Netzarchitektur:}
    \begin{itemize}
        \item Inspiriert durch YOLO (\textbf{Y}ou \textbf{O}nly \textbf{L}ook \textbf{O}nce) \cite{yolov3}
        \item Kann sowohl Klassen als auch Bounding Boxes vorhersagen
        \begin{itemize}
            \item[\rightarrow] Wir brauchen nur Bounding Boxes von Nummernschildern, also Vereinfachung n\"otig
        \end{itemize}
    \end{itemize}
    \textbf{Implementierung:}
    \begin{itemize}
        \item Open Source Deep-Learning Bibliothek Keras
        \item Geschrieben in Python
    \end{itemize}
\end{frame}

\section{OpenCV}

\begin{frame}{OpenCV}
    Was ist OpenCV?
   \begin{itemize}
   \item OpenCV ist eine plattformübergreifende Bibliothek, für Echtzeit-Computer-Vision-Anwendungen
   \item beinhaltet Algorithmen für die Bildverarbeitung und im Rahmen von Computer Vision (CV) auch für maschinelles Lernen
    \end{itemize}
    
   Wofür nutzen wir OpenCV?
   \begin{itemize}
   \item Nutzung für die Verarbeitung des erkannten Nummernschildes (z.B. Tresholding), um die Zeichen besser zu erkennen und richtig auszulesen
   
   \end{itemize}
\end{frame}

\begin{frame}{Beispiel für die Anwendung von OpenCV}

OpenCV wurde bereits auf Nummernschildverarbeitung verwendet:
\begin{figure}
\begin{center}
\includegraphics[scale=0.25]{bilder/Nummer_1.png}
\caption{Original}
\label{Original}
\end{center}
\end{figure}

\begin{figure}
\begin{center}
\includegraphics[scale=0.25]{bilder/Nummer_2_grau.png}
\caption{Graustufen}
\label{Graustufen}
\end{center}
\end{figure}

\end{frame}

\begin{frame}{Beispiel für die Anwendung von OpenCV}

\begin{figure}
\begin{center}
\includegraphics[scale=0.25]{bilder/Nummer_3_treshold.png}
\caption{Tresholding}
\label{Tresholding}
\end{center}
\end{figure}

\begin{figure}
\begin{center}
\includegraphics[scale=0.25]{bilder/Nummer_4_Konturen.png}
\caption{Konturen}
\label{Konturen}
\end{center}
\end{figure}

\end{frame}

\begin{frame}{Beispiel für die Anwendung von OpenCV}

\begin{figure}
\begin{center}
\includegraphics[scale=0.25]{bilder/Nummer_5_Aussortieren.png}
\caption{Aussortierung}
\label{Aussortierung}
\end{center}
\end{figure}

\begin{figure}
\begin{center}
\includegraphics[scale=0.25]{bilder/Nummer_6_SchwarzWeiß.png}
\caption{Schwarze Schrift auf weissem Hintergrund}
\label{SchwarzWeiss}
\end{center}
\end{figure}

Auf das finale Bild (Abbildung 7) wird anschliessend Tesseract angewendet, das die Nummern und Buchstaben ausgibt
\end{frame}

\section{Tesseract}

\begin{frame}{Tesseract}
    \begin{itemize}
    	\item freie Software zur Texterkennung mit vielen vorimplementierten Sprachen
    	\item häufig erprobt für eine Vielzahl von Problemen 
    	% License Plate, Handschrift, Frakturtexte, ...
    \end{itemize}
    \vspace{0.2cm}
    Validierung:
    \begin{align*}
    \frac{\# \text{character} - \# \text{errors}}{\# \text{character}},
    \end{align*}
    wobei $\# \text{character}$ Anzahl der Ziffern im Text und $\# \text{errors}$ Anzahl der fehlerhaft erkannten Ziffern
\end{frame}

\begin{frame}[standout]
  Ausblick:
  \item Definieren der Funktionen 
  \item Zusammenführung der Quellcodes
\end{frame}

\appendix

\begin{frame}[allowframebreaks]{Literatur}

  \bibliography{literatur}
  \bibliographystyle{abbrv}

\end{frame}

\end{document}
